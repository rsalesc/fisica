\documentclass[]{article}
\usepackage{lmodern}
\usepackage{amssymb,amsmath}
\usepackage{ifxetex,ifluatex}
\usepackage{fixltx2e} % provides \textsubscript
\ifnum 0\ifxetex 1\fi\ifluatex 1\fi=0 % if pdftex
  \usepackage[T1]{fontenc}
  \usepackage[utf8]{inputenc}
\else % if luatex or xelatex
  \ifxetex
    \usepackage{mathspec}
  \else
    \usepackage{fontspec}
  \fi
  \defaultfontfeatures{Ligatures=TeX,Scale=MatchLowercase}
\fi
% use upquote if available, for straight quotes in verbatim environments
\IfFileExists{upquote.sty}{\usepackage{upquote}}{}
% use microtype if available
\IfFileExists{microtype.sty}{%
\usepackage{microtype}
\UseMicrotypeSet[protrusion]{basicmath} % disable protrusion for tt fonts
}{}
\usepackage[margin=1in]{geometry}
\usepackage{hyperref}
\hypersetup{unicode=true,
            pdftitle={Física - Discussão Projeto I},
            pdfborder={0 0 0},
            breaklinks=true}
\urlstyle{same}  % don't use monospace font for urls
\usepackage{color}
\usepackage{fancyvrb}
\newcommand{\VerbBar}{|}
\newcommand{\VERB}{\Verb[commandchars=\\\{\}]}
\DefineVerbatimEnvironment{Highlighting}{Verbatim}{commandchars=\\\{\}}
% Add ',fontsize=\small' for more characters per line
\usepackage{framed}
\definecolor{shadecolor}{RGB}{248,248,248}
\newenvironment{Shaded}{\begin{snugshade}}{\end{snugshade}}
\newcommand{\KeywordTok}[1]{\textcolor[rgb]{0.13,0.29,0.53}{\textbf{#1}}}
\newcommand{\DataTypeTok}[1]{\textcolor[rgb]{0.13,0.29,0.53}{#1}}
\newcommand{\DecValTok}[1]{\textcolor[rgb]{0.00,0.00,0.81}{#1}}
\newcommand{\BaseNTok}[1]{\textcolor[rgb]{0.00,0.00,0.81}{#1}}
\newcommand{\FloatTok}[1]{\textcolor[rgb]{0.00,0.00,0.81}{#1}}
\newcommand{\ConstantTok}[1]{\textcolor[rgb]{0.00,0.00,0.00}{#1}}
\newcommand{\CharTok}[1]{\textcolor[rgb]{0.31,0.60,0.02}{#1}}
\newcommand{\SpecialCharTok}[1]{\textcolor[rgb]{0.00,0.00,0.00}{#1}}
\newcommand{\StringTok}[1]{\textcolor[rgb]{0.31,0.60,0.02}{#1}}
\newcommand{\VerbatimStringTok}[1]{\textcolor[rgb]{0.31,0.60,0.02}{#1}}
\newcommand{\SpecialStringTok}[1]{\textcolor[rgb]{0.31,0.60,0.02}{#1}}
\newcommand{\ImportTok}[1]{#1}
\newcommand{\CommentTok}[1]{\textcolor[rgb]{0.56,0.35,0.01}{\textit{#1}}}
\newcommand{\DocumentationTok}[1]{\textcolor[rgb]{0.56,0.35,0.01}{\textbf{\textit{#1}}}}
\newcommand{\AnnotationTok}[1]{\textcolor[rgb]{0.56,0.35,0.01}{\textbf{\textit{#1}}}}
\newcommand{\CommentVarTok}[1]{\textcolor[rgb]{0.56,0.35,0.01}{\textbf{\textit{#1}}}}
\newcommand{\OtherTok}[1]{\textcolor[rgb]{0.56,0.35,0.01}{#1}}
\newcommand{\FunctionTok}[1]{\textcolor[rgb]{0.00,0.00,0.00}{#1}}
\newcommand{\VariableTok}[1]{\textcolor[rgb]{0.00,0.00,0.00}{#1}}
\newcommand{\ControlFlowTok}[1]{\textcolor[rgb]{0.13,0.29,0.53}{\textbf{#1}}}
\newcommand{\OperatorTok}[1]{\textcolor[rgb]{0.81,0.36,0.00}{\textbf{#1}}}
\newcommand{\BuiltInTok}[1]{#1}
\newcommand{\ExtensionTok}[1]{#1}
\newcommand{\PreprocessorTok}[1]{\textcolor[rgb]{0.56,0.35,0.01}{\textit{#1}}}
\newcommand{\AttributeTok}[1]{\textcolor[rgb]{0.77,0.63,0.00}{#1}}
\newcommand{\RegionMarkerTok}[1]{#1}
\newcommand{\InformationTok}[1]{\textcolor[rgb]{0.56,0.35,0.01}{\textbf{\textit{#1}}}}
\newcommand{\WarningTok}[1]{\textcolor[rgb]{0.56,0.35,0.01}{\textbf{\textit{#1}}}}
\newcommand{\AlertTok}[1]{\textcolor[rgb]{0.94,0.16,0.16}{#1}}
\newcommand{\ErrorTok}[1]{\textcolor[rgb]{0.64,0.00,0.00}{\textbf{#1}}}
\newcommand{\NormalTok}[1]{#1}
\usepackage{graphicx,grffile}
\makeatletter
\def\maxwidth{\ifdim\Gin@nat@width>\linewidth\linewidth\else\Gin@nat@width\fi}
\def\maxheight{\ifdim\Gin@nat@height>\textheight\textheight\else\Gin@nat@height\fi}
\makeatother
% Scale images if necessary, so that they will not overflow the page
% margins by default, and it is still possible to overwrite the defaults
% using explicit options in \includegraphics[width, height, ...]{}
\setkeys{Gin}{width=\maxwidth,height=\maxheight,keepaspectratio}
\IfFileExists{parskip.sty}{%
\usepackage{parskip}
}{% else
\setlength{\parindent}{0pt}
\setlength{\parskip}{6pt plus 2pt minus 1pt}
}
\setlength{\emergencystretch}{3em}  % prevent overfull lines
\providecommand{\tightlist}{%
  \setlength{\itemsep}{0pt}\setlength{\parskip}{0pt}}
\setcounter{secnumdepth}{0}
% Redefines (sub)paragraphs to behave more like sections
\ifx\paragraph\undefined\else
\let\oldparagraph\paragraph
\renewcommand{\paragraph}[1]{\oldparagraph{#1}\mbox{}}
\fi
\ifx\subparagraph\undefined\else
\let\oldsubparagraph\subparagraph
\renewcommand{\subparagraph}[1]{\oldsubparagraph{#1}\mbox{}}
\fi

%%% Use protect on footnotes to avoid problems with footnotes in titles
\let\rmarkdownfootnote\footnote%
\def\footnote{\protect\rmarkdownfootnote}

%%% Change title format to be more compact
\usepackage{titling}

% Create subtitle command for use in maketitle
\newcommand{\subtitle}[1]{
  \posttitle{
    \begin{center}\large#1\end{center}
    }
}

\setlength{\droptitle}{-2em}
  \title{Física - Discussão Projeto I}
  \pretitle{\vspace{\droptitle}\centering\huge}
  \posttitle{\par}
  \author{}
  \preauthor{}\postauthor{}
  \date{}
  \predate{}\postdate{}


\begin{document}
\maketitle

\subsubsection{Parâmetros dos
experimentos}\label{parametros-dos-experimentos}

\begin{Shaded}
\begin{Highlighting}[]
\NormalTok{p.massa =}\StringTok{ }\FloatTok{0.2107}

\CommentTok{# Posição inicial do fim da bandeira}
\NormalTok{p.bandeira =}\StringTok{ }\FloatTok{0.204}

\CommentTok{# Tamanho da bandeira}
\NormalTok{p.tam_bandeira =}\StringTok{ }\FloatTok{0.0975}

\CommentTok{# Distância entre sensores adjacentes}
\NormalTok{p.MRU.distancias =}\StringTok{ }\KeywordTok{c}\NormalTok{(}\FloatTok{0.373}\NormalTok{, }\FloatTok{0.322}\NormalTok{, }\FloatTok{0.655}\NormalTok{, }\FloatTok{0.37}\NormalTok{)}
\NormalTok{p.MRU.posicoes =}\StringTok{ }\KeywordTok{cumsum}\NormalTok{(p.MRU.distancias)}


\NormalTok{p.MUV.distancias =}\StringTok{ }\KeywordTok{c}\NormalTok{(}\FloatTok{0.207} \OperatorTok{+}\StringTok{ }\NormalTok{p.bandeira, }\FloatTok{0.414}\NormalTok{, }\FloatTok{0.427}\NormalTok{, }\FloatTok{0.49}\NormalTok{)}
\NormalTok{p.MUV.posicoes =}\StringTok{ }\KeywordTok{cumsum}\NormalTok{(p.MUV.distancias)}

\NormalTok{p.MUV_flag.distancias =}\StringTok{ }\KeywordTok{c}\NormalTok{(}\FloatTok{0.406}\NormalTok{, }\FloatTok{0.49}\NormalTok{, }\FloatTok{0.354}\NormalTok{, }\FloatTok{0.495}\NormalTok{)}
\NormalTok{p.MUV_flag.posicoes =}\StringTok{ }\KeywordTok{cumsum}\NormalTok{(p.MUV.distancias)}
\end{Highlighting}
\end{Shaded}

\subsubsection{Dados dos experimentos}\label{dados-dos-experimentos}

Os dados dos experimentos MU e MUV já foram carregados a partir de
arquivos \texttt{.csv}.

\begin{Shaded}
\begin{Highlighting}[]
\NormalTok{dados.MRU =}\StringTok{ }\KeywordTok{read.csv}\NormalTok{(}\StringTok{"mru.csv"}\NormalTok{)}
\KeywordTok{grid.table}\NormalTok{(dados.MRU)}
\end{Highlighting}
\end{Shaded}

\begin{flushleft}\includegraphics{discussao1_files/figure-latex/unnamed-chunk-2-1} \end{flushleft}

\begin{Shaded}
\begin{Highlighting}[]
\NormalTok{dados.MUV =}\StringTok{ }\KeywordTok{read.csv}\NormalTok{(}\StringTok{"muv.csv"}\NormalTok{)}
\KeywordTok{grid.table}\NormalTok{(dados.MUV)}
\end{Highlighting}
\end{Shaded}

\begin{flushleft}\includegraphics{discussao1_files/figure-latex/unnamed-chunk-3-1} \end{flushleft}

\begin{Shaded}
\begin{Highlighting}[]
\NormalTok{dados.MUV_flag =}\StringTok{ }\KeywordTok{read.csv}\NormalTok{(}\StringTok{"muv_flag.csv"}\NormalTok{)}
\KeywordTok{grid.table}\NormalTok{(dados.MUV_flag)}
\end{Highlighting}
\end{Shaded}

\begin{flushleft}\includegraphics{discussao1_files/figure-latex/unnamed-chunk-4-1} \end{flushleft}

\subsubsection{Funções e constantes}\label{funcoes-e-constantes}

\begin{Shaded}
\begin{Highlighting}[]
\CommentTok{# Constantes comuns (physics)}
\NormalTok{ph.G =}\StringTok{ }\FloatTok{9.807}

\CommentTok{# Funções comuns (physics)}
\NormalTok{ph.peso =}\StringTok{ }\ControlFlowTok{function}\NormalTok{(m, }\DataTypeTok{g =}\NormalTok{ ph.G) \{ m}\OperatorTok{*}\NormalTok{g \}}
\NormalTok{ph.e.potencial =}\StringTok{ }\ControlFlowTok{function}\NormalTok{(m, h, }\DataTypeTok{g =}\NormalTok{ ph.G) \{ m}\OperatorTok{*}\NormalTok{g}\OperatorTok{*}\NormalTok{h \}}
\NormalTok{ph.e.cinetica =}\StringTok{ }\ControlFlowTok{function}\NormalTok{(v, m) \{ }\FloatTok{0.5}\OperatorTok{*}\NormalTok{m}\OperatorTok{*}\NormalTok{v}\OperatorTok{^}\DecValTok{2}\NormalTok{ \}}
\NormalTok{ph.trabalho =}\StringTok{ }\ControlFlowTok{function}\NormalTok{(f, dx, theta) \{ f}\OperatorTok{*}\NormalTok{dx}\OperatorTok{*}\NormalTok{theta \}}

\CommentTok{# Funções MRU}
\NormalTok{MRU.velocidade =}\StringTok{ }\ControlFlowTok{function}\NormalTok{(ds, dt) \{ ds}\OperatorTok{/}\NormalTok{dt \}}
\NormalTok{MRU.deslocamento =}\StringTok{ }\ControlFlowTok{function}\NormalTok{(v, t) \{ v}\OperatorTok{*}\NormalTok{t \}}

\CommentTok{# Funções MUV}
\NormalTok{MUV.velocidade =}\StringTok{ }\ControlFlowTok{function}\NormalTok{(a, t, }\DataTypeTok{v0 =} \DecValTok{0}\NormalTok{) \{ v0 }\OperatorTok{+}\StringTok{ }\NormalTok{a}\OperatorTok{*}\NormalTok{t \}}
\NormalTok{MUV.deslocamento =}\StringTok{ }\ControlFlowTok{function}\NormalTok{(a, t, }\DataTypeTok{v0 =} \DecValTok{0}\NormalTok{) \{ t}\OperatorTok{*}\NormalTok{v0 }\OperatorTok{+}\StringTok{ }\FloatTok{0.5}\OperatorTok{*}\NormalTok{a}\OperatorTok{*}\NormalTok{t}\OperatorTok{^}\DecValTok{2}\NormalTok{ \}}
\end{Highlighting}
\end{Shaded}

\subsubsection{Tabelas}\label{tabelas}

\paragraph{MRU}\label{mru}

\begin{Shaded}
\begin{Highlighting}[]
\NormalTok{velocidades =}\StringTok{ }\KeywordTok{t}\NormalTok{(}\KeywordTok{MRU.velocidade}\NormalTok{(p.MRU.posicoes }\OperatorTok{-}\StringTok{ }\NormalTok{p.bandeira, }\KeywordTok{t}\NormalTok{(dados.MRU)))}
\KeywordTok{colnames}\NormalTok{(velocidades) =}\StringTok{ }\KeywordTok{paste0}\NormalTok{(}\StringTok{"v"}\NormalTok{, (}\DecValTok{1}\OperatorTok{:}\DecValTok{4}\NormalTok{))}

\KeywordTok{grid.table}\NormalTok{(}\KeywordTok{round}\NormalTok{(velocidades, }\DataTypeTok{digits=}\DecValTok{2}\NormalTok{))}
\end{Highlighting}
\end{Shaded}

\begin{flushleft}\includegraphics{discussao1_files/figure-latex/unnamed-chunk-6-1} \end{flushleft}

\begin{Shaded}
\begin{Highlighting}[]
\NormalTok{cineticas =}\StringTok{ }\KeywordTok{ph.e.cinetica}\NormalTok{(velocidades, p.massa)}
\KeywordTok{colnames}\NormalTok{(cineticas) =}\StringTok{ }\KeywordTok{paste0}\NormalTok{(}\StringTok{"ec"}\NormalTok{, (}\DecValTok{1}\OperatorTok{:}\DecValTok{4}\NormalTok{))}

\KeywordTok{grid.table}\NormalTok{(}\KeywordTok{round}\NormalTok{(cineticas, }\DataTypeTok{digits=}\DecValTok{2}\NormalTok{))}
\end{Highlighting}
\end{Shaded}

\begin{flushleft}\includegraphics{discussao1_files/figure-latex/unnamed-chunk-7-1} \end{flushleft}

\paragraph{MUV}\label{muv}

\begin{Shaded}
\begin{Highlighting}[]
\NormalTok{velocidades =}\StringTok{ }\KeywordTok{t}\NormalTok{(}\KeywordTok{MRU.velocidade}\NormalTok{(p.MRU.posicoes }\OperatorTok{-}\StringTok{ }\NormalTok{p.bandeira, }\KeywordTok{t}\NormalTok{(dados.MRU)))}
\KeywordTok{colnames}\NormalTok{(velocidades) =}\StringTok{ }\KeywordTok{paste0}\NormalTok{(}\StringTok{"v"}\NormalTok{, (}\DecValTok{1}\OperatorTok{:}\DecValTok{4}\NormalTok{))}

\KeywordTok{grid.table}\NormalTok{(}\KeywordTok{round}\NormalTok{(velocidades, }\DataTypeTok{digits=}\DecValTok{2}\NormalTok{))}
\end{Highlighting}
\end{Shaded}

\begin{flushleft}\includegraphics{discussao1_files/figure-latex/unnamed-chunk-8-1} \end{flushleft}


\end{document}
